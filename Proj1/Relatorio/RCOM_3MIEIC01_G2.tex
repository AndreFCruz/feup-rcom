\documentclass[a4paper]{article}

%use the english line for english reports
%usepackage[english]{babel}
\usepackage[portuguese]{babel}
\usepackage[utf8]{inputenc}
\usepackage{indentfirst}
\usepackage{graphicx}
\usepackage{verbatim}
\usepackage{fancyhdr}
\usepackage{listings}
\usepackage{color}

\definecolor{dkgreen}{rgb}{0,0.6,0}
\definecolor{gray}{rgb}{0.5,0.5,0.5}
\definecolor{mauve}{rgb}{0.58,0,0.82}

\lstset{frame=tb,
  language=C,
  aboveskip=3mm,
  belowskip=3mm,
  showstringspaces=false,
  columns=flexible,
  basicstyle={\small\ttfamily},
  numbers=none,
  numberstyle=\tiny\color{gray},
  keywordstyle=\color{blue},
  commentstyle=\color{dkgreen},
  stringstyle=\color{mauve},
  breaklines=true,
  breakatwhitespace=true,
  tabsize=3
}

\begin{document}

\setlength{\textwidth}{16cm}
\setlength{\textheight}{22cm}

\title{\Huge\textbf{1º Trabalho Laboratorial:}\linebreak\linebreak
\Huge\textbf{Ligação de Dados}\linebreak\linebreak\linebreak
\Large\textbf{Relatório}\linebreak\linebreak
\linebreak\linebreak
\includegraphics[scale=0.1]{images/feup-logo.png}\linebreak\linebreak
\linebreak
\Large{Mestrado Integrado em Engenharia Informática e Computação} \linebreak\linebreak
\Large{Redes de Computadores}\linebreak
}

\author{\textbf{Turma 1 Grupo 2:}\\
\linebreak\\
André Cruz - 201503776 \\
Bruno Piedade - 201505668 \\
Edgar Carneiro - 201503748 \\
\linebreak\linebreak \\
 \\ Faculdade de Engenharia da Universidade do Porto \\ Rua Roberto Frias, s\/n, 4200-465 Porto, Portugal \linebreak\linebreak
\linebreak\linebreak\vspace{1cm}}

\maketitle
\thispagestyle{empty}

%************************************************************************************************
%************************************************************************************************

\newpage

%Todas as figuras devem ser referidas no texto. %\ref{fig:codigoFigura}
%
%%Exemplo de código para inserção de figuras
%%\begin{figure}[h!]
%%\begin{center}
%%escolher entre uma das seguintes três linhas:
%%\includegraphics[height=20cm,width=15cm]{path relativo da imagem}
%%\includegraphics[scale=0.5]{path relativo da imagem}
%%\includegraphics{path relativo da imagem}
%%\caption{legenda da figura}
%%\label{fig:codigoFigura}
%%\end{center}
%%\end{figure}
%
%
%\textit{Para escrever em itálico}
%\textbf{Para escrever em negrito}
%Para escrever em letra normal
%``Para escrever texto entre aspas''
%
%Para fazer parágrafo, deixar uma linha em branco.
%
%Como fazer bullet points:
%\begin{itemize}
	%\item Item1
	%\item Item2
%\end{itemize}
%
%Como enumerar itens:
%\begin{enumerate}
	%\item Item 1
	%\item Item 2
%\end{enumerate}
%
%\begin{quote}``Isto é uma citação''\end{quote}

%Sumário
\Huge\textbf{Sumário}\linebreak\linebreak

Pimeiro parágrafo sobre o contexto do trabalho.

SEgundo parágrafo sobre as principais conclusões do relatório.
\newpage

%1.Introdução
\section{Introdução}

Objetivos do trabalho e do relatorio? (COPIAR do guiao?)

descrição da lógica do relatório com indicações sobre o tipo de informação que poderá ser encontrada em cada uma secções seguintes
\newpage

%2.Arquitetura
\section{Arquitetura}

Blocos funcionais e interfaces
\newpage

%3.Estrutura do código
\section{Estrutura do Código}

APIs, principais estruturas de dados, principais funções e sua relação com a arquitetura)
\newpage

%4. Casos de uso principais
\section{Casos de uso principais}

(identificação; sequências de chamada de funções)
\newpage

%5. Protocolo de ligação lógica
\section{Protocolo de ligação lógica}

  (identificação dos principais aspectos funcionais; descrição da estratégia de implementação destes aspectos com apresentação de extratos de código)
\newpage

%6. Protocolo de aplicação
\section{Protocolo de aplicação}

  (identificação dos principais aspectos funcionais; descrição da estratégia de implementação destes aspectos com apresentação de extratos de código)
\newpage

%7. Validação
\section{Validação}

   (descrição dos testes efectuados com apresentação quantificada dos resultados, se possível)
\newpage

%8. Eficiência do protocolo de ligação de dados
\section{Eficiência do protocolo de ligação de dados}

   (caraterização estatística da  eficiência do protocolo, feita com recurso a medidas sobre o código desenvolvido. A caracterização teórica de um protocolo Stop\&Wait, que deverá ser usada como termo de comparação, encontra-se descrita nos slides de Ligação Lógica das aulas teóricas). 
\newpage

%9. Conclusões
\section{Eficiência do protocolo de ligação de dados}

  (síntese da informação apresentada nas secções anteriores; reflexão sobre os objectivos de aprendizagem alcançados)
\newpage

%Anexo I
\section{Anexo I}
\begin{lstlisting}[language=C]


\end{lstlisting}

\end{document}
